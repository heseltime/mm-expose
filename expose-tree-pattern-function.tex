%% AMS-LaTeX Created with the Wolfram Language : www.wolfram.com

\documentclass{article}
\usepackage{amsmath, amssymb, graphics, setspace}

\newcommand{\mathsym}[1]{{}}
\newcommand{\unicode}[1]{{}}

\newcounter{mathematicapage}
\begin{document}

\title{Expos{\' e}: A Tree Pattern Function in Mathematica}
\author{For: {``}Wissenschaftliches Arbeiten,{''} Software Engineering at Fachhochschule Ober{\" o}sterreich, Campus Hagenberg.}
\date{}
\maketitle

The present author has implemented the following { }set of Wolfram Language functions in a Wolfram Community project and would like to expand upon
the underlying concepts and the language used (Heseltine, 2023), in an effort to further his own understanding and present a helpful guide to the
language and its functional programming paradigm support.

\begin{doublespace}
\noindent\(\pmb{\text{proofID}[\text{Grid}[\{\_\_\_,\{\text{ID},\text{id$\_$}\},\_\_\_\},\_\_\_]]\text{:=}\text{id};}\\
\pmb{}\\
\pmb{\text{subproofs}[}\\
\pmb{\text{Grid}[}\\
\pmb{\{\_\_\_,\{\text{Proofs},\text{OpenerView}[\{\text{Arguments},\text{Column}[\text{subproofs$\_$},\_\_\_]\},}\\
\pmb{\_\_\_]\},\_\_\_\},\_\_\_]]\text{:=}\text{subproofs};}\\
\pmb{\text{subproofs}[\text{proof$\_$}]\text{:=}\{\};}\\
\pmb{}\\
\pmb{\text{getLeanTree}[\text{proof$\_$}]\text{:=}}\\
\pmb{\text{Tree}[\text{proofID}[\text{proof}],\text{getLeanTree}\text{/@}\text{subproofs}[\text{proof}],}\\
\pmb{\text{TreeElementLabelStyle}\text{-$>$}}\\
\pmb{\text{All}\text{-$>$}\text{Directive}[\text{White},16,\text{FontFamily}\text{-$>$}\text{{``}Times New Roman{''}}],}\\
\pmb{\text{TreeElementStyle}\text{-$>$}}\\
\pmb{\text{All}\text{-$>$}\text{Directive}[\text{EdgeForm}[\text{Black}],\text{RGBColor}[\text{{``}$\#$B6094A{''}}]]]}\)
\end{doublespace}

An expos{\' e} of the concepts follows, but first an introduction of both the problem space and language will be helpful. The complete code base
is available as a G $\unicode{0008}$itHub repository.

The Problem Space

The objective is to extract a tree data structure in the form of certain integer mathematical proof IDs and the related children IDs from a grid
expression in Wolfram Language. The grid is interpreted by Mathematica but comes from an external program called LEAN, an automated and interactive
theorem prover. LEAN and the particular implementation details of processing the tree can be treated as a black box for the purposes of this study.

The Wolfram Language

Wolfram Research (2023) describes the language on a high level like this: {``}The Wolfram Language is a highly developed knowledge-based language
that unifies a broad range of programming paradigms and uses its unique concept of symbolic programming to add a new level of flexibility to the
very concept of programming.{''} Also, on the point of functional programming: {``}Functional programming is a highly developed and deeply integrated
core feature of the Wolfram Language, made dramatically richer and more convenient through the symbolic nature of the language.{''}

The functional aspects of Wolfram Language to the point needed to understand the function of interest will be of particular import here, but less
so the symbolic side. Mathematica is the concrete programming environment used to explore the concepts with examples.

\section*{Lists and Replacements}

Going back to our object of interest:

\begin{doublespace}
\noindent\(\pmb{\text{proofID}[\text{Grid}[\{\_\_\_,\{\text{ID},\text{id$\_$}\},\_\_\_\},\_\_\_]]\text{:=}\text{id};}\)
\end{doublespace}

The set of functions we are studying use lists (demarcated with curly brackets, here in a nested fashion) in constructing a pattern, but lists can
also be generated: Lists are central, general objects in the Wolfram Language since they can be made to represent other objects. \\
\\
Internally, they are, of course, functions.

\begin{doublespace}
\noindent\(\pmb{\text{FullForm}[\{1,2,3\}]}\)
\end{doublespace}

\begin{doublespace}
\noindent\(\text{List}[1,2,3]\)
\end{doublespace}

Starting with list creation utilities, here are some helpful functions for handling lists.

\begin{doublespace}
\noindent\(\pmb{\text{Range}[3]}\)
\end{doublespace}

\begin{doublespace}
\noindent\(\{1,2,3\}\)
\end{doublespace}

This is the same as:

\begin{doublespace}
\noindent\(\pmb{\text{Range}[1,3,1]}\)
\end{doublespace}

\begin{doublespace}
\noindent\(\{1,2,3\}\)
\end{doublespace}

Compare this to Table, which takes a Wolfram Language iterator in the form \textit{ $\{$i, imin, imax, step$\}$}:

\begin{doublespace}
\noindent\(\pmb{\text{Table}[2k, \{k,1,10,2\}]}\)
\end{doublespace}

\begin{doublespace}
\noindent\(\{2,6,10,14,18\}\)
\end{doublespace}

As with Range, \textit{ imin} and \textit{ step} can be omitted (set to 1).

One might like to change the display orientation, here using {``}$\%${''} to access the evaluation immediately prior:

\begin{doublespace}
\noindent\(\pmb{\text{Column}[\%]}\)
\end{doublespace}

\begin{doublespace}
\noindent\(\begin{array}{l}
 2 \\
 6 \\
 10 \\
 14 \\
 18 \\
\end{array}\)
\end{doublespace}

Matrices are similarly a matter of display, of nested lists. A construction using two iterators might look as follows.

\begin{doublespace}
\noindent\(\pmb{\text{Table}[i * j, \{i,1,3\},\{j,1,4\}] \text{//} \text{MatrixForm}}\)
\end{doublespace}

\begin{doublespace}
\noindent\(\left(
\begin{array}{cccc}
 1 & 2 & 3 & 4 \\
 2 & 4 & 6 & 8 \\
 3 & 6 & 9 & 12 \\
\end{array}
\right)\)
\end{doublespace}

It is important to note the difference to this expression reversing the order of the iterators.

\begin{doublespace}
\noindent\(\pmb{\text{Table}[i * j,\{j,1,4\},\{i,1,3\}] \text{//} \text{MatrixForm} \text{(* // TableForm *)}}\)
\end{doublespace}

\begin{doublespace}
\noindent\(\left(
\begin{array}{ccc}
 1 & 2 & 3 \\
 2 & 4 & 6 \\
 3 & 6 & 9 \\
 4 & 8 & 12 \\
\end{array}
\right)\)
\end{doublespace}

That is, the first iterator determines the number of rows, in the 2D context. Another way to say this is that the inner iterator determines the value
of the outer iterator. The mathematical expression being evaluated is, of course, commutative, however.

For measuring lists, the Length-Function gives the outermost dimension, whereas \textit{ Dimensions[]} gives all dimensions of nested lists. The
number of dimensions is given by \textit{ ArrayDepth[]}.\\
\\
For testing and sampling lists the following functions are available. \textit{ Position[list, element] }gives the (list of) index positions at which
\textit{ element }sits, whereas \textit{ Select[list, predicate] }tests elements of the list against \textit{ predicate, }sampling those elements
that satisfy, say, even parity with \textit{ EvenQ. }Given a position, elements can be extracted using \textit{ Part[] }or its short form in double-square-brackets
notation, here as a matrix-example.

\begin{doublespace}
\noindent\(\pmb{\text{Table}\left[a_{i,j}, \{i,3\},\{j,3\}\right] \text{//} \text{MatrixForm}}\)
\end{doublespace}

\begin{doublespace}
\noindent\(\left(
\begin{array}{ccc}
 a_{1,1} & a_{1,2} & a_{1,3} \\
 a_{2,1} & a_{2,2} & a_{2,3} \\
 a_{3,1} & a_{3,2} & a_{3,3} \\
\end{array}
\right)\)
\end{doublespace}

\begin{doublespace}
\noindent\(\pmb{\%[[2,1]]}\)
\end{doublespace}

\begin{doublespace}
\noindent\(a_{2,1}\)
\end{doublespace}

\textit{ Take[] }is similar to \textit{ Part[] }but allows sampling consecutively placed elements in a list.

This section ends with an analysis of the {``}$\_${''}-character to symbolize any other Mathematica object (Wolfram Language expressions), which
is useful for patterns in general, and particularly, the proofID function at the beginning of this section. The crucial difference is between two
underscores, {``}$\_\_${''} or BlankSequence, standing for any sequence of one or more expressions, and three underscores, {``}$\_\_\_${''} or BlankNullSequence,
which also allows zero such expressions.

The following example, taken from Wolfram Research (2023), makes the limitation of BlankSequence clear.

\begin{doublespace}
\noindent\(\pmb{f[\text{x$\_\_$}]\text{:=}\text{Length}[\{x\}]}\)
\end{doublespace}

\begin{doublespace}
\noindent\(\pmb{\{f[x,y,z],f[]\}}\)
\end{doublespace}

\begin{doublespace}
\noindent\(\{3,f[]\}\)
\end{doublespace}

Contrast this with use of the BlankNullSequence:

\begin{doublespace}
\noindent\(\pmb{\text{f2}[\text{x$\_\_\_$}]\text{:=}p[x,x]}\)
\end{doublespace}

\begin{doublespace}
\noindent\(\pmb{\{\text{f2}[],\text{f2}[1],\text{f2}[1,a]\}}\)
\end{doublespace}

\begin{doublespace}
\noindent\(\{p[],p[1,1],p[1,a,1,a]\}\)
\end{doublespace}

Blank ({``}$\_${''}) is the most limited pattern character in that it can stand for any Wolfram Language Expression, but only exactly one such expression.

\begin{doublespace}
\noindent\(\pmb{\text{f3}[\text{x$\_$}] \text{:=} \text{{``}evaluated{''}}}\)
\end{doublespace}

\begin{doublespace}
\noindent\(\pmb{\{\text{f3}[], \text{f3}[1], \text{f3}[1, \text{a}]\}}\)
\end{doublespace}

\begin{doublespace}
\noindent\(\{\text{f3}[],\text{evaluated},\text{f3}[1,\text{a}]\}\)
\end{doublespace}

\section*{Functional Programming}

With lists there is now a background in place to consider Wolfram Language{'}s functional aspects in more detail. For a definition, I refer to the
Mathematica Cookbook (Mangano, 2010, p. 31): {``}All functional languages emphasize the evaluation of expressions to produce values rather than commands
or statements that are executed for their side effects.{''} Since Mathematica also allows for functions like \textit{ Do[],} the language is not
purely functional, instead supporting at least also the procedural paradigm. The focus here however is functional programming by a given example,
not a comparison of paradigms or more analysis of the functional one.

With this, the following core functional programming {``}primitives{''} are summed up in a helpful reference table in (Mangano, 2010, p. 26) and
will lead to a discussion of the relative idioms and their application.

\begin{doublespace}
\noindent\(\pmb{\text{Grid}[\{\{\text{{``}Function{''}},\text{{``}Operator{''}},\text{{``}Description{''}}\},}\\
\pmb{\{\text{{``}Map[f, expr]{''}},\text{{``}/@{''}},}\\
\pmb{\text{$\texttt{"}$Return the list that results from executing f on each }}\\
\pmb{\text{element of an expr$\texttt{"}$}\}, }\\
\pmb{\{\text{{``}Apply[f, expr]{''}}, \text{{``}@@{''}}, }\\
\pmb{\text{$\texttt{"}$Return the result of replacing the ehad of a list with }}\\
\pmb{\text{function f$\texttt{"}$}\}, }\\
\pmb{\{\text{{``}Apply[f, expr, $\{$1$\}$]{''}}, \text{{``}@@@{''}}, }\\
\pmb{\text{$\texttt{"}$Applies f at level 1 inside list. In other words, }}\\
\pmb{\text{replace the head of all elements$\texttt{"}$}\}, }\\
\pmb{\{\text{{``}Fold[f, x, $\{$a1, a2, a3$\}$]{''}}, \text{{``}N/A{''}}, }\\
\pmb{\text{$\texttt{"}$If list hast length 0, return x, otherwise return }}\\
\pmb{\text{f[f[f[x, a1], a2], a3]...$\texttt{"}$}\}, }\\
\pmb{\{\text{{``}FoldList[f, x, $\{$a1, a2, a3, ...$\}$]{''}}, \text{{``}N/A{''}}, }\\
\pmb{\text{{``}Return the list $\{$x, f[x, a1], f[f[x, a1], a2], ...$\}${''}}\}, }\\
\pmb{\{\text{{``}Nest[f, expr, n]{''}}, \text{{``}N/A{''}}, }\\
\pmb{\text{$\texttt{"}$Return the list f[f[f[...f[expr]...]]] (i.e. f applied }}\\
\pmb{\text{n times)$\texttt{"}$}\}, \{\text{{``}NestList[f, expr, n]{''}}, \text{{``}N/A{''}}, }\\
\pmb{\text{$\texttt{"}$Return the list $\{$x, f[expr], f[f[expr]], ...$\}$ where }}\\
\pmb{\text{f repeats up to n times$\texttt{"}$}\}\},\text{Frame}\text{-$>$}\text{All}]}\)
\end{doublespace}

\begin{doublespace}
\noindent\(\begin{array}{ccc}
 \text{Function} & \text{Operator} & \text{Description} \\
 \text{Map[f, expr]} & \text{/@} & \text{Return the list that results from executing f on each element of an expr} \\
 \text{Apply[f, expr]} & \text{@@} & \text{Return the result of replacing the ehad of a list with function f} \\
 \text{Apply[f, expr, $\{$1$\}$]} & \text{@@@} & \text{Applies f at level 1 inside list. In other words, replace the head of all elements} \\
 \text{Fold[f, x, $\{$a1, a2, a3$\}$]} & \text{N/A} & \text{If list hast length 0, return x, otherwise return f[f[f[x, a1], a2], a3]...} \\
 \text{FoldList[f, x, $\{$a1, a2, a3, ...$\}$]} & \text{N/A} & \text{Return the list $\{$x, f[x, a1], f[f[x, a1], a2], ...$\}$} \\
 \text{Nest[f, expr, n]} & \text{N/A} & \text{Return the list f[f[f[...f[expr]...]]] (i.e. f applied n times)} \\
 \text{NestList[f, expr, n]} & \text{N/A} & \text{Return the list $\{$x, f[expr], f[f[expr]], ...$\}$ where f repeats up to n times} \\
\end{array}\)
\end{doublespace}

This also covers almost all of the sort-hand notation after blanks, where ampersand ({``}$\&${''}) and hash ({``}$\#${''}) still need mentioning:
this ties in perfectly with the topic of functional operations and goes to show that indeed, functional is at the heart of the language, since the
short-hand notation offered caters to the paradigm. This example illustrates the usages, in so-called pure functions (lambda expressions), starting
with the named-function counter-example.

\begin{doublespace}
\noindent\(\pmb{h[\text{x$\_$}]\text{:=}f[x]+g[x]}\)
\end{doublespace}

Combine this with \textit{ Map[]} in the following way.

\begin{doublespace}
\noindent\(\pmb{\text{Map}[h,\{a,b,c\}]}\)
\end{doublespace}

\begin{doublespace}
\noindent\(\{f[a]+g[a],f[b]+g[b],f[c]+g[c]\}\)
\end{doublespace}

The pure-function alternative is written as follows.

\begin{doublespace}
\noindent\(\pmb{\text{Map}[f[\#]+g[\#]\&,\{a,b,c\}]}\)
\end{doublespace}

\begin{doublespace}
\noindent\(\{f[a]+g[a],f[b]+g[b],f[c]+g[c]\}\)
\end{doublespace}

Here ampersand demarcates the end of the function expression, and hash marks the arguments. The non-short-hand version is:

\begin{doublespace}
\noindent\(\pmb{\text{Map}[\text{Function}[x,f[x]+ g[x]],\{a,b,c\}]}\)
\end{doublespace}

\begin{doublespace}
\noindent\(\{f[a]+g[a],f[b]+g[b],f[c]+g[c]\}\)
\end{doublespace}

I refer to the Functional Operators Tutorial (Wolfram Research, 2023) for the minor details including, for example, the modifications of the hash
operator, such as \textit{ $\#$n} or \textit{ $\#\#$}. 

\pmb{ TODO: functional programming idioms (map is one, see cookbook), application to the example being studied. Finish reading cookbook chapter and
correlate in the Wellin book. Idioms also in the tutorial. Idiom in the example?}

\section*{Rule-based Programming}

TODO

\section*{Recursion}

TODO

\section*{Limitations and Summary}

This study did not cover numerics, graphics programming, { }front end programming, or writing packages, all major topics related to the current discussion:
An Introduction to Programming with Mathematica (2005, Wellin et al.) completes the picture at the intended overview level of analysis. The focus
here has been on one particular example, solving one problem, in Wolfram Language; it is involved enough to be viewed in context of the major themes
chosen after introducing the language in broad strokes. These themes were lists; functional programming as considered for itself but also relative
to other paradigms; rule-based programming as the particularly helpful approach in this situation; and recursion, which we could not have easily
done without.

\section*{References}

Heseltine, J. (2023) \textit{ Empirical metamathematics: extending the Lean-to-Mathematica bridge. }Wolfram Community. Available at: https://community.wolfram.com/groups/-/m/t/2957419
(Accessed: 26 July 2023).\\
Mangano, S. (2010) \textit{ Mathematica Cookbook. }Sebastopol, CA: O{'}Reilly Media, Inc.\\
Wellin, P. R. (2005) \textit{ An Introduction to Programming in Mathematica. }Cambridge, UK: Cambridge University Press.\\
Wolfram Research (2023) Wolfram Language $\&$ System Documentation Center. Available at: https://reference.wolfram.com/language/ (Accessed: 23 July
2023).

\end{document}
